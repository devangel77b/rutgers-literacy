\documentclass{article}

\title{Module 10: Literacy and student learning standards}
\author{Dennis Evangelista}
\date{\today}

\usepackage[round]{natbib}

\begin{document}
\maketitle

\begin{abstract}
This module assignment consists of a reflection on the readings from Module 2 (What is literacy?) and consideration of how this course has prepared me to address NJ Student Learning Standards for literacy in my classroom. I found one Module 2 reading very discouraging; the other less so, and place the value in Module 2 in finding my own answers for literacy in terms of the power gained by my students and the broader view of literacy (for example, science and STEM literacy). 
\end{abstract}

\section{Part A: What is literacy? }
%(1) Discuss your reading experience with each Module II Reading. Which was more accessible? Explain what criteria you used to decide which you found easier/more difficult to comprehend.
%(2) Reflect on the strategies you used for comprehension, i.e. re-reading, note-taking, paragraph summaries, paraphrasing thoughts, self-questioning, looking up unfamiliar vocabulary, etc...
%(3) Why do you think the assignment called for you to read only 2 ½ pages of one article versus the other? 

There were two Module 2 readings; \citet{gee89} and \citet{campbell90}. \cite{gee89} was obtuse and technical, spending many pages to develop fiddly, technical definitions of certain concepts. \cite{campbell90} was written as if the author was speaking to the audience directly, using a Socratic discourse to advance the author's view. The result was that \cite{gee89} made me want to claw my eyes out, while I was more receptive to the messages in \cite{campbell90}. Perhaps this is why the assignment called for us to read only two and a half pages of \cite{gee89}. 

I am a scientist. In my field, we approach technical literature by reading the abstract, section headings, introduction (and especially the research question and any testable hypotheses proposed by the authors), and discussion. We also pay special attention to the graphs, figures and tables; and any captions, to ask ourselves if the data support the authors' conclusions. This left me at a disadvantage in comprehending both \cite{gee89} and \cite{campbell90}, as neither presents testable hypotheses or data. 

\cite{gee89} advances arguments by flowery language of who sounds more scholarly or erudite. On the other hand, \cite{campbell90} at least grounds arguments in terms of what definitional choices might mean for, e.g. a student in Western Australia, a job seeker, or a scholar reading articles on subjects (the Big Bang) outside their normal expertise. This sort of anecdotal information is the closest either of these readings got to ``data''. The authors took a while to get to their respective points, so I found myself having to re-read and paraphrase/simplify for myself. I did not see unfamiliar vocabulary but \cite{gee89} did try to give many shades of meaning and split technical fine hairs which I found annoying. 

I recall when I first did the Module 2 assignment, I was turned off by the ``ivory tower'' nature of especially \cite{gee89}. On another round of re-reads, I am still discouraged from thinking about this subject by \cite{gee89}. \cite{campbell90} was easier to comprehend; but even more powerful for me was being challenged to explain what literacy was in my own words; and having to comment on it from the point of view of literacy in science and STEM, which I have come to understand as both being well- and widely read in science literature but also being conversant in the ways of thinking, arguing, and understanding the world and the work of other scientists. I tell my STEM students that they do the hard work so they will have power over things like AI and technology in the future; and literacy is that same kind of power. 





\section{Part B: How the course has prepared me to meet standards for literacy}
%Respond in 75-100 words to the following question: How has this course prepared you to address the New Jersey Student Learning Standards for literacy in your classroom?
At first I thought this course would not be very useful for me. I teach AP Physics C for 11th and 12th grade students in a STEM magnet program. They can all read, write, and speak English perfectly well. I had a poor opinion coming in from a course that had advertised to tell me how to incorporate mathematics in my class, which really was a total waste. 

On second reflection on the NJ Student Learning Standards \cite{njsls}, in the areas of English Language Arts, I see all those same words I use in a science class in trying to force my students to think, form testable hypotheses, rationally test them, and reach logical conclusions. I see the same appeal to have these ideas and be able to effectively communicate them to other human beings, who can then understand and value my students thinking and comprehension. It has been useful to have a place to discuss how I can encourage these skills, as ``soft skills'' are all the rage in STEM right now and being an effective explainer and communicator of science has always been appreciated in certain fields. 

The course has given me some strategies and ideas. Currently, I am discouraged because my Phase I section is heavily weighted for elementary/middle school non STEM; also it is mainly special ed. My students need something somewhat different. The format of this course (primarily asynchronous, primarily self-directed) has provided me with freedom to seek those things that I feel would be more relevant in my practice, based on experiences trying to fill in the missing skills in college-bound STEM students and early undergraduates. When I was teaching undergraduates, the holes I saw were in describing their scientific work in writing, in effectively explaining the reason behind engineering design decisions, in writing with accompanying figures and tables and data, and in orally communicating with posters and conference talks. In addition, undergraduates were bad at surfing through peer reviewed scientific literature and making sense of the state of a field. The strategies they had been taught in reading and comprehending the Great Gatsby, for example, were not helping them in these areas. With a broader understanding of what is meant by ``literacy'' I feel more confident pushing these areas in my classes; that I am not infringing on what my humanities colleagues are doing but rather enhancing. I look forward to putting these into practice. 








\bibliographystyle{plainnat}
\bibliography{module-10.bib}
\end{document}
